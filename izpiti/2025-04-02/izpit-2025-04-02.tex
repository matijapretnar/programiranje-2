\documentclass[11pt]{../izpit}
\usepackage{fouriernc}
\usepackage{xcolor}
\usepackage[outputdir=../]{minted}
\usemintedstyle{xcode}
\begin{document}

\izpit[ucilnica=P01, naloge=-1]{Programiranje 2: 1. pisni izpit}{2.\ april 2025}{
  Čas reševanja je 60 minut.
  Veliko uspeha!
}

%%%%%%%%%%%%%%%%%%%%%%%%%%%%%%%%%%%%%%%%%%%%%%%%%%%%%%%%%%%%%%%%%%%%%%%

\naloga[\tocke{10}]

Za vsakega izmed spodnjih programov prikažite vse spremembe sklada in kopice, če poženemo funkcijo \mintinline{rust}{main}. Za vsako spremembo označite, po kateri vrstici v kodi se zgodi.

\podnaloga
\begin{minted}[linenos]{rust}
fn main() {
    let mut vec = vec![1, 2, 3];
    for i in &vec {
        println!("{}", i);
    }
    vec.push(4);
}
\end{minted}

\podnaloga
\begin{minted}[linenos]{rust}
fn f(c: usize) -> usize {
    println!("{c}");
    c * 10
}

fn main() {
    let x = String::from("Hello");
    let n = f(x.len());
    let y = x;
    println!("{} {}", y, n);
}
\end{minted}

\podnaloga
\begin{minted}[linenos]{rust}
fn f(x: usize, s: String) -> usize {
    let y = x + s.len();
    g(y)
}

fn g(z: usize) -> usize {
    z * 2
}

fn main() {
    let s = String::from("Hello");
    let result = f(5, s);
    println!("{}", result);
}
\end{minted}

\podnaloga
\begin{minted}[linenos]{rust}
fn f(s: &String) {
    let c = s.clone();
    println!("Copy: {}", c);
}

fn g(mut s: String) -> String {
    s.push_str(" World");
    f(&s);
    s
}

fn h() -> String {
    let wrapped = String::from("Hello");
    g(wrapped)
}

fn main() {
    let result = h();
    println!("Final String: {}", result);
}

\end{minted}


%%%%%%%%%%%%%%%%%%%%%%%%%%%%%%%%%%%%%%%%%%%%%%%%%%%%%%%%%%%%%%%%%%%%%%%

\naloga[\tocke{20}]

Definirajmo tip \mintinline{rust}{Ali<T, U>}, kot najbolj enostaven vsotni tip. Dopolnite signaturo spodnje implementacije pri čemer naj bodo tipi čim bolj splošni (morebitne odločitve utemeljite). Če v dani prostor ni treba dopisati ničesar, ga prečrtajte.

{\large
\begin{minted}{rust}
enum Ali<T, U> {
    Levi(T),
    Desni(U),
}

impl<T, U> Ali<T, U> {
    fn je_levi(_____ self) -> _____ bool {
        // Preveri, ali smo v varianti `levo`
    }

    fn zamenjaj(_____ self) -> _____ Ali<_____, _____> {
        // Zamenja varianti.
    }

    // Ena od teh, obe sta ok
    fn dobi_desnega(_____ self) -> _____ Option<__________> {
        // Vrne vrednost v desni varianti (če je to mogoče).
    }

    fn map_levi<F, V>(_____ self, f: _____ F) -> __________
      where
        F: __________,
    {
        // Levo vrednost preslika s funkcijo `f`
    }
    
    fn zamenjaj_levega(_____ self, t: _____ T) {
        // Na mestu zamenja levo varianto
    }

    fn uporabi<F, G, V, Z>(self, f: F, g: G) -> _____ Ali<V, Z>
      where
        F: __________ -> _____ Ali<V, Z>,
        G: __________ -> _____ Ali<V, Z>,
    {
        // Uporabi ustrezno funkcijo na ustrezni varianti
        // let a: Ali<usize, usize> = Ali::Desni(5);
        // let f1 = |x| Ali::Levi(x + 1);
        // let f2 = |x| Ali::Desni(vec![0;x]);
        // let k = a.uporabi(f1, f2);
        // println!("{:?}", k.dobi_desnega()); // Some([0,0,0,0])
    }
}
\end{minted}
}


%%%%%%%%%%%%%%%%%%%%%%%%%%%%%%%%%%%%%%%%%%%%%%%%%%%%%%%%%%%%%%%%%%%%%%%

\naloga[\tocke{20}]
Za vsakega izmed spodnjih programov:
\begin{enumerate}
  \item razložite, zakaj in s kakšnim namenom Rust program zavrne;
  \item program popravite tako, da bo veljaven in bo učinkovito dosegel prvotni namen.
\end{enumerate}

\podnaloga
\begin{minted}{rust}
fn print_length(s: String) {
    println!("Length: {}", s.len());
}
fn main() {
    let text = String::from("Rust");
    print_length(text);
    println!("{}", text);
}
\end{minted}

\podnaloga
\begin{minted}{rust}
fn main() {
    let mut data = vec![1, 2, 3];
    let first = &data[0];
    data.push(4);
    println!("{}", first);
}
\end{minted}

\podnaloga
\begin{minted}{rust}
fn main() {
    let s = String::from("hello");
    let r = &s;
    let t = s;
    println!("{}", r);
}
\end{minted}

\podnaloga
\begin{minted}{rust}
fn f() -> &String {
    let s = String::from("hello");
    &s
}

fn main() {
    let r = f();
    println!("{}", r);
}
\end{minted}

\podnaloga
\begin{minted}{rust}
struct O { x: Vec<i32>, }
impl O {
    fn a(&self) -> &i32 { &self.x[0] }
    fn b(self) -> Vec<i32> { self.x }
}
// Cilj je izpisati vsebino vektorja in prvi element kar se da učinkovito.
// Smiselen popravek je lahko tudi v definiciji ali implementaciji strukture O.
fn main() {
    let o = O { x: vec![1, 2, 3] };
    let first = o.a();
    println!("{:?}", o.b());
    println!("{}", first);
}
\end{minted}


\end{document}
